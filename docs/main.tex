\documentclass[paper=a4,fontsize=12pt,titlepage]{scrartcl}
%-----------------------------------------------
\usepackage[utf8]{inputenc}
\usepackage[T1]{fontenc}
\usepackage[french]{babel}
\usepackage{lmodern}
%-----------------------------------------------
\usepackage{amsmath,amssymb,amsthm,mathtools,stmaryrd}
%-----------------------------------------------
\usepackage{xcolor}
\usepackage{graphicx}
\usepackage{tikz, pgf}
\usepackage{float}
\usepackage{adjustbox}

\usetikzlibrary{arrows,shapes,positioning}
%-----------------------------------------------
\usepackage[ruled,lined]{algorithm2e}
\usepackage{listings}
%-----------------------------------------------
\usepackage[
	colorlinks=true,
	breaklinks=true,
	linkcolor=blue,
	filecolor=magenta,
	urlcolor=cyan,
	pdftitle={Project},
	pdfsubject={Project}
]{hyperref}
%-----------------------------------------------
\usepackage{typearea}

\KOMAoptions{
	DIV=12,
	BCOR=0mm,
	abstract=true,
	pagesize=auto
}
\recalctypearea
%-----------------------------------------------
\newenvironment{AThreeLandscape}{%
	\clearpage
	\KOMAoptions{paper=a3, paper=landscape}%
	\recalctypearea%
	\thispagestyle{plain}%
}{%
	\clearpage
	\KOMAoptions{paper=a4, paper=portrait}%
	\recalctypearea%
}
%-----------------------------------------------
\newenvironment{AFourLandscape}{%
	\clearpage
	\KOMAoptions{paper=a4, paper=landscape}%
	\recalctypearea%
	\thispagestyle{plain}%
}{%
	\clearpage
	\KOMAoptions{paper=a4, paper=portrait}%
	\recalctypearea%
}
%-----------------------------------------------
\newtheorem{theorem}{Théorème}
\newtheorem{axiom}{Axiome}

\theoremstyle{definition}
\newtheorem{mydef}{Définition}
\newtheorem{mycor}{Corollaire}

\theoremstyle{remark}
\newtheorem{myrem}{Remarque}

\theoremstyle{plain}
\newtheorem{myprop}{Proposition}

\DeclareMathOperator{\Ima}{Im}
\DeclareMathOperator{\Ker}{Ker}
\DeclareMathOperator{\Card}{Card}
\DeclareMathOperator{\Sp}{Sp}
\DeclareMathOperator{\Inf}{Inf}
\DeclareMathOperator{\Vect}{Vect}
\newcommand{\R}{\mathbb{R}}
\newcommand{\N}{\mathbb{N}}
\newcommand{\B}{\mathbb{B}}
\newcommand{\C}{\mathbb{C}}
\newcommand{\K}{\mathbb{K}}
\newcommand{\Mat}{\mathcal{M}at}

\DeclarePairedDelimiter\abs{\lvert}{\rvert}%
\DeclarePairedDelimiter\norm{\lVert}{\rVert}%

\makeatletter
\let\oldabs\abs
\def\abs{\@ifstar{\oldabs}{\oldabs*}}
%
\let\oldnorm\norm
\def\norm{\@ifstar{\oldnorm}{\oldnorm*}}
\makeatother
%-----------------------------------------------
\sloppy
\hyphenpenalty=10000
%-----------------------------------------------
\title{
	\Huge Rasende Roboter\\
	\Large Projet IA41
}

\author{
	Noa \textsc{FOUICH}\\
	Corentin \textsc{HAUTEFAYE}\\
	William \textsc{LE GALLOU}\\
	Kenneth \textsc{SOARES}
}

\date{Janvier 2026}
%-----------------------------------------------
\begin{document}
	
	\maketitle
	
	\renewcommand*{\abstractname}{Introduction}
	\begin{abstract}
		\thispagestyle{plain}
		Dans le cadre de l'UE IA41 (\textit{Semestre A25}), nous avons réalisé une implémentation du jeu de décision \textit{Rasende Roboter}. L'objectif de ce projet est
		de programmer un système capable de fournir une solution aux configurations proposées, autrement dit de permettre à un utilisateur de jouer des parties contre l'ordinateur.
	\end{abstract}
	
	\tableofcontents
	\clearpage
%-----------------------------------------------
	\section{Présentation générale}
	\subsection{Rappel de l'énoncé du sujet}
	
	\textbf{INSERT TEXT HERE}
	
	\subsection{Conventions}
	
	\textbf{INSERT TEXT HERE}
	
	\subsection{Outils utilisés}
	
	Pour réaliser ce projet, les outils suivants ont été utilisés:
	\begin{itemize}
		\item \textbf{Python 3} 
		\item \textbf{PyGame} pour l'interface graphique
		\item \textbf{Git} et \textbf{GitHub} pour le contrôle des versions
		\item \textbf{\LaTeX} pour la rédaction du rapport
	\end{itemize}
%-----------------------------------------------
	\newpage
	\section{Spécification du problème}
	
	Dans cette section, on cherche à généraliser et formaliser une partie du jeu. 
	
	\subsection{Notations}
	
	Soit $k\in\mathbb{N}\backslash\{0;1\}$. Soit $n\in\mathbb{N}$ tel que $n\ge k+1$. Afin de représenter le plateau de jeu, on définit les notations suivantes:
	\begin{itemize}
		\item $B=\llbracket 1;n\rrbracket^2$ une grille de $n\times n$ cases.
		\item $R=\{r_i\;|\; i\in\llbracket 1;k\rrbracket\}$ l'ensemble des $k$ robots où pour tout $i\in\llbracket 1;k\rrbracket, r_i=(x_i,y_i)\in B$.
		\item $C=\left\{\{(x,y),(x',y')\}\;\Big|\; \left\{\begin{array}{ll}
					(x,y) &\in B \\
					(x',y') &\in B \\
					\abs{x-x'}+\abs{y-y'} &=1
				\end{array}\right.\right\}$ l'ensemble des collisions entre deux cases de $B$.
		\item $D=\left\{(1,0),(-1,0),(0,1),(0,-1)\right\}$ l'ensemble des directions cardinales.
	\end{itemize}
	
	\subsection{Espace d'états}
	
	\begin{mydef}
		On définit un \textit{état du jeu} par tout $k$-uplet de $B$.
		L'ensemble de tous les états possibles, noté $\mathcal{S}$, correspond à l'ensemble des positions occupées par les $k$ robots, d'où $\mathcal{S}=B^k$.
	\end{mydef}
	
	Il est important de noter que de ce fait, on obtient $\Card(\mathcal{S})=n^{2k}$. Or, la définition proposée n'impose pas à un état d'être valide. En effet, il est impossible que deux robots occupent la même case, d'où:
	
	\begin{mydef}
		On définit \textit{l'ensemble des états valides}, ou \textit{espace d'états}, et on note $\mathcal{S}_v=\left\{(r_1,\ldots,r_k)\in B^k\;\Big|\;\forall(i,j)\in\llbracket 1;k\rrbracket^2, (i\ne j)\implies (r_i\ne r_j)\right\}$. Un élément de $\mathcal{S}_v$ est dit \textit{valide}.
	\end{mydef}
	
	L'inclusion $\mathcal{S}_v\subset\mathcal{S}$ est ainsi triviale.
	
	\begin{myprop}
		On a $\Card\left(\mathcal{S}_v\right)=\frac{(n^2)!}{(n^2-k)!}$.
	\end{myprop}
	
	\begin{proof}
		On cherche le nombre de $k$-uplets injectifs $(r_1,\ldots,r_k):\llbracket 1;k\rrbracket\rightarrow B$. Formellement, chaque fonction $f:\llbracket 1;k\rrbracket\rightarrow B$ qui est injective correspond à un état valide. Or, le nombre de fonctions injectives de $\llbracket 1;k\rrbracket$ dans $B$ est exactement le nombre d'arrangements de $k$ parmi $n^2$ éléments.
		
		D'où, la conclusion.
	\end{proof}
	
	\begin{myrem}
		On obtient ainsi le fait que l'espace d'états croît très rapidement en fonction de $n$ et $k$.
	\end{myrem}
	
	\subsection{Fonction de transition}
	
	\begin{mydef}
		On définit la \textit{fonction de transition} pour passer d'un état valide à un autre, d'où:
		
		\[
			\begin{array}{lll}
				\delta :& \mathcal{S}_v\times\llbracket 1;k\rrbracket\times D &\rightarrow\mathcal{S}_v \\
				&((r_1,\ldots,r_k),i,d)&\mapsto (r_1,\ldots,r_i+dt,\ldots,r_k)
			\end{array}
		\]
		
		avec $t=\min \Big\{ s \in \mathbb{N}^* \;\Big|\; \begin{array}{ll}
			&(x_i,y_i) + d(s+1) \in R \\
			\text{ou} &\{ (x_i,y_i) + ds, (x_i,y_i) + d(s+1) \} \in C 
		\end{array}\Big\}$
	\end{mydef}
	
	\begin{myrem}
		Comme pour le passage d'un état à un autre, il est facile de trouver les variables $i$ et $d$, il convient de simplifier l'écriture de $\delta(S,i,d)$ par $\delta(S)$.
	\end{myrem}
	
	\begin{mydef}
		Soit $t\in\llbracket 1;k\rrbracket$ et $S_0=(r_1,\ldots,r_k)\in\mathcal{S}_v$ l'état initial. Soit $g\in B$ une case. On dit que $g$ est \textit{atteignable} depuis $S_0$ par $r_t$ \underline{ssi} il existe une suite finie de $m+1\in\mathbb{N}$ états $(S_i)_{i\in\llbracket 0;m\rrbracket}$ telle que pour tout $i\in\llbracket 1;m\rrbracket, \delta(S_{i-1})=S_i$ et $(S_m)_t=g$ où $(S_m)_t$ désigne la position du $t$-ième robot dans l'état considéré.
	\end{mydef}
	
	On définit ainsi l'ensemble $G$ des cases objectifs jouables par:
	\[
		G=\left\{g\in B \;|\;\exists m\in\mathbb{N}, \exists (S_i)_{i\in\llbracket 0;m\rrbracket}, \left\{\begin{array}{ll}
			\forall i\in\llbracket 1;m\rrbracket, &\delta(S_{i-1})=S_i\\
			(S_m)_t=g
		\end{array}\right.\right\}
	\]
	
	\begin{mydef}
		On appelle \textit{configuration de jeu} tout sextuplet
		\[
		(k, n, t, C, R, g)
		\]
		où :
		\begin{itemize}
			\item $k$ est le nombre de robots,
			\item $n$ est la taille du plateau,
			\item $t\in\llbracket 1;k\rrbracket$ est l'indice du robot cible,
			\item $R$ est l'ensemble des positions initiales des robots,
			\item $C$ est l'ensemble des obstacles entre cases adjacentes,
			\item $g \in G$ est la case objectif du robot cible.
		\end{itemize}
	\end{mydef}
	
%-----------------------------------------------
	\newpage
	\section{Analyse du problème}
	\subsection{Nature et caractéristiques du problème}
	\subsection{Contraintes et difficultés}
	\subsection{Structures de données envisageables}
	\subsection{Approches algorithmiques possibles}
	\subsubsection{Recherche non informée}
	\subsubsection{Recherche informée}
	\subsubsection{Critères de choix d'une approche}
%-----------------------------------------------
	\newpage
	\section{Traitement du problème}
	\subsection{Principe général de la solution retenue}
	\subsection{Algorithmes implémentés}
	\subsubsection{Parcours en largeur d'abord}
	\subsubsection{Algorithme A*}
	\subsubsection{Heuristique utilisée}
	\subsection{Organisation du programme}
%-----------------------------------------------
	\newpage
	\section{Études de cas \& Résultats}
	\subsection{Description des situations testées}
	\subsection{Résultats obtenus}
	\subsection{Interprétation des résultats}
%-----------------------------------------------	
	\newpage	
	\section{Retour d'expérience}
	\subsection{Organisation}
	\subsection{Difficultés rencontrées}
	\subsection{Suggestions d'amélioration}
%-----------------------------------------------
	\newpage	
	\section{Conclusion}
\end{document}
